\documentclass{article}
\usepackage{amsmath}
\usepackage[utf8]{inputenc}
\usepackage[bulgarian]{babel}

\title{Курсова работа №4}
\author{Александър Игнатов \\ Ф№ 62136 }
\date{\today}

\begin{document}

\maketitle

\section*{Условие}

\[
    (8x+3)y'' + y' = 0,\, y(0) = 1,\, y'(0) = -2
\]

\section*{Решение}

Предполагаме, че търсената функция и нейните производни имат следния вид на степенен ред:

\begin{align*}
    y(x)   &= \sum_{n=0}^{\infty} y_n x^n \\
    y'(x)  &= \sum_{n=1}^{\infty} y_n n x^{n-1} \\
    y''(x) &= \sum_{n=2}^{\infty} y_n n(n-1) x^{n-2}
\end{align*}

Имаме \( y(0) = 1 \) и \( y'(0) = -2 \), следователно

\begin{align*}
    y(x)  &= 1 + \sum_{n=1}^{\infty} y_n x^n \\
    y'(x) &= -2 + \sum_{n=2}^{\infty} y_n n x^{n-1}
\end{align*}

Така уравнението от условието добива вида

\begin{gather*}
    (8x+3)\sum_{n=2}^\infty y_n n(n-1)x^{n-2} + \sum_{n=2} ^\infty y_n n x^{n-1} - 2 = 0 \\
    \sum_{n=1}^\infty [8n(n+1) + 1]y_{n+1}x^n + \sum_{n=0}^\infty 3(n+2)(n+1)y_{n+2}x^n - 2 = 0
\end{gather*}

При \( n = 0 \) имаме
\begin{gather*}
    3 . 2 . 1 . y_2 x^0 - 2 = 0 \\
    y_2 = \frac{1}{3}
\end{gather*}

Следователно чрез премахване на нулевия член на сумата:
\begin{gather*}
    \sum_{n=0}^\infty 3(n+2)(n+1)y_{n+2}x^n = 2 + \sum_{n=1}^\infty 3(n+2)(n+1)y_{n+2}x^n
\end{gather*}

Уравнението от условието придобива вида:

\begin{gather*}
    \sum_{n=1}^\infty \{[8n(n+1) + 1]y_{n+1}x^n + 3(n+2)(n+1)y_{n+2}\}x^n = 0
\end{gather*}

При \( n \geq 1 \):

\begin{gather*}
    [8n(n+1) + 1]y_{n+1} + 3y_{n+2}(n+2)(n+1) = 0
\end{gather*}

Т.е., за \( n \geq 2 \)

\begin{gather*}
    [8n(n+1) + 1]y_{n} + 3y_{n+1}(n+2)(n+1) = 0 \\
    y_{n+1} = \frac{8n(n+1)+1}{3(n+2)(n+1)} y_n
\end{gather*}

Използвайки разлагане стигаме до крайния отговор.

\[
    y_n = - \frac{6}{7} \bigg(\frac{8}{3}\bigg)^n \binom{\frac{7}{8}}{n}
\]

\section*{Отговор}

\[
    y(x) = 1 - \frac{6}{7}\sum_{n=1}^{\infty} \bigg(\frac{8}{3}\bigg)^n \binom{\frac{7}{8}}{n} x^n
\]

\end{document}