\documentclass{article}
\usepackage{amsmath}
\usepackage[utf8]{inputenc}
\usepackage[bulgarian]{babel}

\title{Курсова работа №4}
\author{Александър Игнатов \\ Ф№ 62136 }
\date{\today}

\begin{document}

\maketitle

\section*{Условие}

\[
    (8x+3)y'' + y' = 0,\, y(0) = 1,\, y'(0) = -2
\]

\section*{Решение}

Предполагаме, че търсената функция и нейните производни имат следния вид на степенен ред:

\begin{align*}
    y(x)   &= \sum_{n=0}^{\infty} y_n x^n \\
    y'(x)  &= \sum_{n=1}^{\infty} y_n n x^{n-1} \\
    y''(x) &= \sum_{n=2}^{\infty} y_n n(n-1) x^{n-2}
\end{align*}

Имаме \( y(0) = 1 \) и \( y'(0) = -2 \), следователно

\begin{align*}
    y(x)  &= 1 + \sum_{n=1}^{\infty} y_n x^n \\
    y'(x) &= -2 + \sum_{n=2}^{\infty} y_n n x^{n-1}
\end{align*}

Така уравнението от условието добива вида

\begin{gather*}
    (8x+3)\sum_{n=2}^\infty y_n n(n-1)x^{n-2} + \sum_{n=2} ^\infty y_n n x^{n-1} - 2 = 0 \\
    \sum_{n=1}^\infty (8n+1)(n+1)y_{n+1}x^n + \sum_{n=0}^\infty 3(n+2)(n+1)y_{n+2}x^n - 2 = 0
\end{gather*}

При \( n = 0 \):
\begin{gather*}
    3 . 2 . 1 . y_2 x^0 = 0 \\
    y_2 = 0
\end{gather*}

При \( n = 1 \):
\begin{gather*}
    32y_2 + 18y_3 - 2 = 0 \\
    y_3 = \frac{1}{9}
\end{gather*}

При \( n \geq 2 \):

\begin{gather*}
    (8n + 1)(n+1)y_{n+1} + 3y_{n+2}(n+2)(n+1) = 0 \\
    y_{n+2} = -\frac{8n+1}{3(n+2)} y_{n+1} \\
    y_{n+2} = \left(-\frac{8n+1}{3(n+2)}\right) \left(-\frac{8n-7}{3(n+1)}\right) \ldots y_3
\end{gather*}

Т.е., за \( n \geq 4 \):

\begin{gather*}
    y_{n} = \left(-\frac{8n-7}{3n}\right) \left(-\frac{8n-15}{3(n-1)}\right) \ldots \left(-\frac{1}{3.4}\right).\frac{-4}{3}.\frac{2}{2}
\end{gather*}

Следователно общият член има формула

\begin{gather}
    y_n = \frac{-6(-1)^{n-3}(8n-7)(8n-15)...}{3^{n-3}n!}
\end{gather}

И търсената функция е

\begin{gather}
    y(x) = 1 - 2x + \frac{1}{9}x^3 - 6\sum_{n=4}^{\infty} \frac{(-1)^{n-3}(8n-7)(8n-15)...}{3^{n-3}n!} x^n
\end{gather}

\section*{Радиус на сходимост}

Радиусът на сходимост \( R \) намираме чрез

\begin{align*}
    R &= \lim_{n \to \infty} \left\vert \frac{a_n}{a_{n+1}} \right\vert = \\
    &= \lim_{n \to \infty} \frac{3^{n-2}(n+1)!(8n-7)(8n-15)...}{3^{n-3}n!(8n+1)(8n-7)(8n-15)...} = \\
    &= \lim_{n \to \infty} \frac{3n + 3}{8n + 1} = \frac{3}{8}
\end{align*}

Следователно за \( \left(-\infty, -\frac{3}{8}\right) \cup \left(\frac{3}{8}, +\infty\right) \) редът е разходящ, а за \( \left( -\frac{3}{8}, \frac{3}{8} \right) \) е абсолютно сходящ.

За точка \( x = -\frac{3}{8} \):

\begin{align*}
    &\, \sum_{n=4}^{\infty}\frac{(-1)^{n-1}(8n-7)(8n-15)...}{3^{n-3}n!}.\frac{(-1)^n3^n}{8^n} = \\
    &= -27\sum_{n=4}^{\infty}\frac{(8n-7)(8n-15)...}{8^nn!} = \\
    &= -27\sum_{n=4}^{\infty} b_n
\end{align*}

Проверка за сходимост на реда \( {b_n} \) по Даламбер:

\begin{align*}
    & \lim_{n \to \infty}\frac{b_{n+1}}{b_n} = \\
    &= \lim_{n \to \infty} \frac{8^nn!(8n+1)(8n-7)(8n-15)...}{8^{n+1}(n+1)!(8n-7)(8n-1)} = \\
    &= \lim_{n \to \infty} \frac{8n+1}{8n+8} = 1 - 0
\end{align*}

С критерия на Даламбер не можем да определим дали редът е сходящ или не, затова трябва да приложим критерият на Раабе-Дюамел:

\begin{align*}
    &\lim_{n \to \infty} n\left( \frac{8n+8}{8n+1} - 1 \right) = \\
    &= \lim_{n \to \infty} \frac{7n}{8n+1} = \frac{7}{8} < 1
\end{align*}

Следователно, редът е разходящ за \( x = -\frac{3}{8} \)

За точка \( x = \frac{3}{8} \) чрез критерият на Раабе-Дюамел и достатъчното условие на Лайбниц намиране, че редът е сходящ. Той в тази точка не е абсолютно сходящ, следователно за \( x = \frac{3}{8} \) редът е условно сходящ.

\end{document}