\documentclass{article}
\usepackage{amsmath}
\usepackage[utf8]{inputenc}
\usepackage[bulgarian]{babel}

\title{Курсова работа №4}
\author{Александър Игнатов \\ Ф№ 62136 }
\date{\today}

\begin{document}

\maketitle

\section*{Условие}

\[
    (8x+3)y'' + y' = 0,\, y(0) = 1,\, y'(0) = -2
\]

\section*{Решение}

Търсената функция и нейните производни имат следния вид на степенен ред:

\begin{align*}
    y(x) &= \sum_{n=0}^{\infty} y_n x^n \\
    y'(x) &= \sum_{n=1}^{\infty} y_n n x^{n-1} \\
    y''(x) &= \sum_{n=2}^{\infty} y_n n(n-1) x^{n-2}
\end{align*}

Имаме \( y'(0) = -2 \), следователно

\[
    y'(x) = -2 + \sum_{n=2}^{\infty} y_n n x^{n-1}
\]

Така уравнението от условието добива вида

\begin{gather*}
    (8x+3)\sum_{n=2}^\infty y_n n(n-1)x^{n-2} + \sum_{n=2} ^\infty y_n n x^{n-1} - 2 = 0 \\
    \sum_{n=1}^\infty [8n(n+1) + 1]y_{n+1}x^n + \sum_{n=0}^\infty 3y_{n+2}(n+2)(n+1)x^n - 2 = 0
\end{gather*}

При \( n = 0 \) имаме
\begin{gather*}
    3y_2 . 2 . 1 - 2 = 0 \Longleftrightarrow y_2 = \frac{1}{3}
\end{gather*}

При \( n \geq 1 \):

\begin{gather*}
    [8n(n+1) + 1]y_{n+1} + 3y_{n+2}(n+2)(n+1) - 2 = 0
\end{gather*}

\section*{Отговор}

\[
    y(x) = 1 - \frac{6}{7}\sum_{n=1}^{\infty} \bigg(\frac{8}{3}\bigg)^n \binom{\frac{7}{8}}{n} x^n
\]



\end{document}